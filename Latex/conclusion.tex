\section{Conclusions}\label{sec:conclusions}

In this article we study the relation between performance score given by lighthouse and the energy consumption of a web app. We designed an experiment with performance score as a factor and three different treatment levels according to the performance category. After conducting the experiments, our results show that web apps with a poor performance score consumes more energy than web apps with good or average performance. Also, the energy consumption for web apps with good performance is significantly smaller than those with average performance; nonetheless, this difference is not as large when compared toe the difference with poor performance web apps. \newline 

Looking into the obtained results, we can see how lighthouse can be used in a promising way to measure and assess the consumed energy. Lighthouse can not only assess, but also provides guidance on how to improve critical performance aspects. Considering this, we recommend the developers to use this tool to set development milestones to improve the web app's performance. In doing so,  developers will increase the likelihood of lowering the energy consumption of their web app, as showed by this research.
\newline

Apart from that, this experiment could be extended performing an analysis on the score of each of the audits used to give a performance score. This way it would be possible to know which audit affects more to the energy consumption and developers could focus on its improvement to reduce energy consumption. \newline

Another possible extension to the study could be done by using different bench-marking tools. In this way, it would be possible to know which tool gives a more precise score to estimate energy consumption.
