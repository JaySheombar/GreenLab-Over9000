\section{Discussion}
From the results of the statistical tests performed in the previous section we can elaborate on our research question. 
The outcome of the Kruskal Wallis test shows that we can reject our null hypothesis. Therefore, we can say that there is a difference in the means of energy consumption of the different performance levels. \newline

Given that Kruskal Wallis is an omnibus statistical test, the nature of these differences is best explained with the Dunn test. With the results from the Dunn test we can see that there is no significant differences of energy consumption between the good and average performance levels of web apps. However, the test shows that the difference in energy consumption of web apps with poor performance levels is quite relevant.
\newline

Finally we try to see the degree of overlap between two distributions of performance levels. With Cliff's Delta test we know that the effect size of a poor performance level against good and average levels is moderate as described by the Cliff's delta guidelines. However, comparing good and average performance levels yields negligible effect sizes. 
\newline

\hlcyan{With the help of data analysis and the obtained results from our experiment we can answer our \textit{research question: To what extent does the performance score of lighthouse correlate with energy consumption in the context of web applications?}}
\newline

\hlcyan{We can conclude that performance score of lighthouse can be used as an indication of how much energy a certain web app consumes. More precisely, we can say that when a web app has a performance score between 0 and 44, the energy consumption is going to be higher than a web app with more than 44 as performance score. }
\newline

\hlcyan{This information can be useful for the web developers who are interested in reducing the energy consumption from their web applications. While developing web apps they can take into account the performance score from lighthouse's performance audit as a guide to estimate how much energy is going to be consumed by that particular web app. And based on this score, they can adopt changes or optimize the codes so that their web app consumes less energy.}


